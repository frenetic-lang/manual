% !TEX root = frenetic_programmers_guide.tex

\chapter{Learning Switch}

\section{Design}
\label{l2_learning_switch:design}

Layer 2, or L2, switching revolutionized networking in the 1990's.
As LAN traffic grew, hub performance rapidly degraded as collisions became more and more frequent.
L2 switches effectively cut the LAN into logical segments, performing the forwarding between them.
This dramatically reduced the number of collisions, and also cut down on the traffic that individual
NIC's had to filter out.  Just as the Plain Old 
Telephone System evolved from party lines to direct lines, LAN hardware evolved from Hubs to 
L2 switches, improving security, speed and signal quality.

Of course, L2 switches were more technically sophisticated than hubs.  They required a processor, memory, and 
some form of software.  In order to know which segments to forward traffic, they needed to 
watch the Ethernet MAC addresses of traffic and theremember their respective ports.  In other words, the switch 
\emph{learns} the MAC-to-port mappings, and thus L2 switches are called learning switches.    

We can simulate the L2 switch with Frenetic.  By doing so, as well see in 
Section \ref{l2_learning_switch:timeouts},
we can add features to the switch with just a little extra programming effort.  At a high-level, 
you can think of a Frenetic network application as:

$$ netkat = f( nib, env ) $$

Where $f$ is your application, $nib$ is the Network Information Base -- the information you have dynamically determined in your network through
packets received by \netkat{pkt_in} -- and
$env$ is other information (fixed configuration files, out-of-band network measurements, or whatever you want).  
The output, $netkat$ is the NetKAT program.

Naturally, $nib$ is critical to a good design.  You don't need to capture all aspects of the network,
only those needed to properly form switch rules.  In an L2 switch, we are really only interested in three 
pieces of data in a packet:

\begin{itemize}
\item The source MAC address
\item The destination MAC address
\item The switch port connected to the host with a particular MAC address
\end{itemize}

So that's all the data we want to capture for the NIB.  Here's a rough design for how we want the switch to behave:

\begin{minted}{python}
if port_for_mac(EthSrc) == None:
  learn(EthSrc, Port)
if port_for_mac(EthDst) != None:
  pkt_out(payload, port)
else
  pkt_out(payload, all_ports_except(port))
\end{minted}

Admittedly this is pretty sketchy, but it covers the interesting cases.  In particular, it covers
Ethernet broadcasts to MAC address ff:ff:ff:ff:ff:ff just by the fact that a source MAC will never
equal ff:ff:ff:ff:ff:ff.  And flooding is exactly what you want to do in that case.

So our NIB must maintain at least a list of MAC-to-port mappings.  
In our Repeater app, our NIB was a single instance variable in the application itself:
\netkat{self.ports}, which held a list of connected ports on the switch.  
Now we'll evolve a little.
In what will become a standard
part of our network apps, we'll model the NIB as a separate object class in Python.  
The following code is in \codefilename{l2_learning_switch/network_information_base.py}:

\inputminted{python}{code/l2_learning_switch/network_information_base.py}

That 
encapsulates the state in one place, making it easy to change underlying data structures later.
It also separates the NIB details from the NetKAT details, making it easier to reuse the NIB
in other applications.  

\section{A First Pass}

One of the problems with our switch pseudocode design is it doesn't fit our notions of NetKAT very well.
NetKAT programs do not have variables, so they can't remember MAC-to-port mappings on their own.
So it appears that every packet must pass through the controller so we can make decisions.
Processing every single packet through the controller clearly violates the First NetKAT principle, but
we can leave that aside for now.  It'll be instructive to build an easy but inefficient L2 switch first.

The following code is in \codefilename{l2_learning_switch/learning1.py}:

\inputminted[linenos]{python}{code/l2_learning_switch/learning1.py}

There are a couple of new details to note:

\begin{itemize}
\item The \netkat{__init__} constructor must call the superclass constructor to properly initialize.
\item Because we are writing in classes, we now distinguish the main loop of this application with 
a check on \netkat{__main__}.  
\item We are using the Frenetic \python{Packet} object discussed in Section \ref{introduction:packet_in}
\item The \netkat{packet_in} looks almost exactly like our pseudocode design
\end{itemize}

Starting up Mininet, Frenetic and our application respectively, we try a \netkat{pingall} in Mininet and see
the following on the console:

\begin{minted}{console}
vagrant@frenetic:~/manual/programmers_guide/code/l2_learning_switch$ python learning1.py
Starting the tornado event loop (does not return).
2016-04-14 12:49:17,228 [INFO] Connected to Frenetic - Switches: {1: [4, 2, 1, 3]}
2016-04-14 12:49:17,229 [INFO] Learning: 9a:0f:ec:39:54:f5 attached to ( 1 )
2016-04-14 12:49:17,258 [INFO] Learning: be:3f:5a:90:8a:ac attached to ( 2 )
2016-04-14 12:49:17,303 [INFO] Learning: 3a:a4:6b:e6:24:25 attached to ( 3 )
2016-04-14 12:49:17,343 [INFO] Learning: f2:a7:c0:cb:90:23 attached to ( 4 )
\end{minted}

The switch works perfectly!  But it's a huge violation of Principle 1: all the traffic 
goes through the controller.  

\section{A More Efficient Switch}

Once we've learned a MAC-to-port
mapping, we shouldn't have to go to the controller for packets destined for that MAC.  The switch
should handle it by itself.

This is actually pretty straightforward.  If we know that MAC 11:11:11:11:11:11 is on port 2, we
can handle it with the following NetKAT program:

\begin{minted}{python}
Filter(EthDstEq("11:11:11:11:11:11")) >> SetPort(2)
\end{minted}

And we just need one of these rules for each MAC we've learned.  But all of these rules are non-overlapping
because they involve different values for \netkat{EthDst}.  So we just Union them all together and that's
our entire NetKAT program.

So let's write some methods for calculating the policies
We'll add this code to learning1 (listed in \codefilename{l2_learning_switch/learning2.py}):

\inputminted[firstline=23,lastline=31]{python}{code/l2_learning_switch/learning2.py}

Note here that \python{(mac, port) = mac_port} unpacks the tuple \python{mac_port} into two variables
\python{mac} and \python{port}.

When shall we install these rules?   We could install them on every incoming packet, but that's a little
overkill. We really only need to recalculate them when we see a newly learned MAC and port.  So we add them
to that conditional:

\inputminted[firstline=41,lastline=43]{python}{code/l2_learning_switch/learning2.py}

Now run it and try a pingall from Mininet:

\begin{minted}{console}
vagrant@frenetic:~/manual/programmers_guide/code/l2_learning_switch$ python learning1.py
Starting the tornado event loop (does not return).
2016-04-14 13:33:22,965 [INFO] Connected to Frenetic - Switches: {1: [2, 4, 1, 3]}
2016-04-14 13:33:26,447 [INFO] Learning: 86:d8:df:f0:95:75 attached to ( 1 )
2016-04-14 13:33:26,453 [INFO] Learning: 4a:1c:9e:9b:50:7c attached to ( 2 )
... STOP
\end{minted}

Uh oh.  Why did we only learn the first two ports?   Let's look at the Frenetic console for a clue:

\begin{minted}{console}
[DEBUG] Installing policy
drop |
(filter ethDst = 4a:1c:9e:9b:50:7c; port := 2 |
 filter ethDst = 86:d8:df:f0:95:75; port := 1)
[DEBUG] Setting up flow table
+----------------------------------------+
| 1 | Pattern                | Action    |
|----------------------------------------|
| EthDst = 4a:1c:9e:9b:50:7c | Output(2) |
|----------------------------------------|
| EthDst = 86:d8:df:f0:95:75 | Output(1) |
|----------------------------------------|
|                            |           |
+----------------------------------------+
\end{minted}

Can you see the problem?  There's no longer a rule to send packets to the controller.  If packets
are destined for the first two MAC addresses, that's not a problem.  But if they're destined for
other MAC addresses, it \emph{is} a problem.

If that's true, why didn't it stop after the first packet?  Remember NetKAT Principle 4, which 
states there is a lag time between rule sending and rule installation.  In this case:

\begin{enumerate}
\item The first packet came to the controller, causing a rule regeneration.  These rules
are sent to the switch, but are not installed yet.
\item The second packet came to the controller, causing a second rule regeneration.
\item The rules from the first packet are installed on the switch, effectively shutting off 
any more packets from going to the controller.
\item The rules from the second packet are installed.  
\end{enumerate}

One thing that definitely \emph{won't} work is to add the following rule with a Union:

\begin{minted}{python}
id >> SendToController("learning_app")
\end{minted}

The \netkat{id} filter matches all packets, and therefore overlaps every other rule.  Even if we place this
rule as the last rule in a set of Unions, \emph{that does not guarantee it'll be fired last.}  Frenetic 
does not guarantee the OpenFlow rules will follow the order of the NetKAT rules.     

There are a few ways to solve this problem, but we'll try an easy one first.  
In Chapter 2, we mentioned briefly that for every \netkat{FieldEq} NetKAT predicate, there is a corresponding
\netkat{FieldNotEq} predicate.  We can use that in our policy, as we see in \netkat{learning3.py}:

\inputminted[firstline=30,lastline=37]{python}{code/l2_learning_switch/learning3.py}

Basically, we want to see all packets with an unfamiliar Ethernet source MAC (because we want to learn the
port) or destination MAC (because we need to flood it out all ports).  Although we could set up 
NetKAT policies to handle the latter case, it tends to be overkill since an unlearned destination MAC
usually replies soon afterwards, the MAC is learned, and everything is good.  

Notice how the this fairly simple NetKAT policy becomes a large OpenFlow table:

\begin{minted}{console}
[DEBUG] Setting up flow table
+------------------------------------------------------+
| 1 | Pattern                | Action                  |
|------------------------------------------------------|
| EthDst = 00:00:00:00:00:01 | Output(1)               |
| EthSrc = 00:00:00:00:00:01 |                         |
|------------------------------------------------------|
| EthDst = 00:00:00:00:00:01 | Output(1)               |
| EthSrc = 00:00:00:00:00:02 |                         |
|------------------------------------------------------|
| EthDst = 00:00:00:00:00:01 | Output(1)               |
| EthSrc = 00:00:00:00:00:03 |                         |
|------------------------------------------------------|
| EthDst = 00:00:00:00:00:01 | Output(1)               |
| EthSrc = 00:00:00:00:00:04 |                         |
|------------------------------------------------------|
| EthDst = 00:00:00:00:00:01 | Output(Controller(128)) |
|------------------------------------------------------|
...
\end{minted}

In fact, the table will have $n^2$ entries where $n$ is the number of learned MACs.  It's a lot easier
to write the NetKAT rules than all these OpenFlow rules.  

\section{Timeouts and Port Moves}
\label{l2_learning_switch:timeouts}

Our learning switch works fine if MAC-to-port assignments never change.  But a network is usually
more fluid than that:

\begin{itemize}
\item Users unplug a host from one port and plug it into another.  In our application, packets will
continue to go to the old port.
\item Users replace one host (and associated MAC) with another host in the same port.  In our application, the
old MAC will continue to take up rule space on the switch, making it more confusing to debug. 
\end{itemize}

Fortunately, plugging and unplugging hosts sends OpenFlow events \netkat{port_up} and \netkat{port_down},
respectively.  We can write hooks that control MAC learning and unlearning.  The following
code is in \netkat{learning4.py}:

\inputminted[firstline=65,lastline=74]{python}{code/l2_learning_switch/learning4.py}

When we make a port change, we call \netkat{update()} to recalculate and send the NetKAT rules down to the 
switch.  This keeps the forwarding tables in sync with the NIB.  

If we can rely on \netkat{port_up} and \netkat{port_down} events, this approach would work fine.
However, in the real world, the following things can happen:

\begin{itemize}
\item The \netkat{port_up} or \netkat{port_down} message might not fire.  Since they rely on power being
present or absent, they are not always reliable.
\item The messages might arrive in the wrong order, as in the port ``flip-flopping'' between active and 
inactive status.
\end{itemize}

Modern switches solve these issues by holding MAC-to-port mappings for a certain amount of time, then 
timing them out and (if they're still connected) relearning them.  Pure OpenFlow flow rules emulate
this by assigning a timeout value to each flow rule.   But NetKAT doesn't have timeouts, and indeed since
NetKAT policies don't map one-to-one to flow table rules, it'd be difficult to pass this information on.

But our application obeys the following principle:

\setcounter{principle}{4}

\begin{principle}
Do not rely on network state.  Always assume the current packet is the first one you're seeing.   
\end{principle}
 
That means we can restart the application at any time.  This clears out the NIB and sets the flow table
back to its initial ``send all packets to controller'' state.  
There will be a slight performance penalty as MAC-to-port mappings are relearned, but eventually the
NIB will be filled wth all current MAC addresses \ldots and no outdated ones.

Following this principle yields another benefit: fault tolerance.  If the switch loses connectivity with 
our application, the switch will continue to function.  No new MACs will be learned, but otherwise the 
network will continue to run.  When the application returns, it will start with a clear NIB and relearn MACs.  

In other words, a fully populated NIB is not critical to switch operation.  It makes things much faster
by providing the basis for the NetKAT rules.  But it's not so important that it needs to be persisted or
shared, making the overall design much simpler.  

\section{Summary}

Our learning switch application is a mere 130 lines of code, but it does a lot:

\begin{itemize}
\item Mac addresses are learned in the NIB as packets arrive at the switch
\item The flow table is updated to match the NIB
\item Broadcast packets are automatically forwarded to all ports
\item It handles hosts moved to other ports or replaced with other devices
\item It is fault tolerant, and can easily tolerate restarts
\end{itemize}

The latest traditional switches can do even more.  For example, you can plugin a Wireless Access Port or 
WAP to a switch port.  Practically, the WAP acts like a multiplexer allowing multiple MACs to 
attach to a single port.   It turns out our application handle multiple MACs mapped to a port, although it's 
difficult to model this in Mininet.  We can enforce certain security constraints, like the number of MACs on
a particular WAP, simply by changing the NIB class.

It's this kind of flexibility that makes SDN an important evolutionary move.  In the next chapter, we'll
inject some more intelligence into the network to handle Virtual LANs, or VLANs. 