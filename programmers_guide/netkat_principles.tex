% !TEX root = frenetic_programmers_guide.tex

\chapter{NetKAT Principles}

\newtheorem{principle}{Principle}

\section{Efficient SDN}

In a nutshell, the \texttt{packet\_in()} hook receives network packets and the \texttt{pkt\_out()} command sends
network packets.  
In theory, you could use these two to implement arbitrarily-complex network clients and servers.  
You could build switches and routers, but also HTTP servers, Email servers, Database servers, or any other 
network server.  

That said, you probably wouldn't want to.
OpenFlow and Frenetic are optimized for small, very selective packet inspections and creations.  
The more packets you inspect through \texttt{packet\_in()}, the slower your controller will be, and
the more likely that packets will be dropped or sent out of sequence.  

\begin{principle}
\label{principle:controller}
Keep as much traffic out of the controller as possible.
Instead, program NetKAT policies to make most of the decisions inside the switch.  
\end{principle}

So let's go back to our naive Repeater application:

\begin{lstlisting}
import sys
sys.path.append('../src/frenetic/lang/python')
import frenetic
from frenetic.syntax import *

class RepeaterApp(frenetic.App):

    def connected(self):
        self.update( id >> SendToController("repeater_app") )

    def packet_in(self, dpid, port_id, payload):
        out_port = 2 if port_id = 1 else 1
        self.pkt_out(dpid, payload, [ Send(out_port_id) ] )

app = RepeaterApp()
app.start_event_loop()
\end{lstlisting}

Here, \emph{every single packet} goes from the switch to Frenetic to the net app and back out. 
That's horribly inefficient, and it unecessarily so since all the decisions can be made inside the switch.
So let's write a more efficient one:

\begin{lstlisting}
import sys
sys.path.append('../src/frenetic/lang/python')
import frenetic
from frenetic.syntax import *

class RepeaterApp2(frenetic.App):

    def connected(self):
        policy_port_one = Filter(PortEq(1)) >> Send(2)
        policy_port_two = Filter(PortEq(2)) >> Send(1)
        self.update( policy_port_one | policy_port_two )

app = RepeaterApp()
app.start_event_loop()
\end{lstlisting}

Wow!  
That program takes principle \ref{principle:controller} very seriously, to the point where \emph{no} packets 
arrive at the controller.
All of the configuration of the switch is done up front.

The \texttt{Filter(PortEq(1)) >> Send(2)} policy is a pretty common pattern in NetKAT.
You first whittle down the incoming flood of packets to a certain subset with \texttt{Filter} and a 
predicate.
Then you apply a policy or series of policies, \texttt{Send} being the most popular.
We'll look at combining policies in the section~\ref{section:combining}.

If you've worked with OpenFlow, you might wonder how the NetKAT rules get translated to OpenFlow rules.
In this example, it's fairly straightforward.
You get two OpenFlow rules in the rule table, which you can see in the Frenetic debug window:

TODO

But this is not true in general.  
One NetKAT rule may expand into many, many OpenFlow rules.
And it may go the opposite direction to: where different NetKAT rules are combined to create one OpenFlow rule.
It's the same thing that happens with most compiled languages -- the rules that govern the compiled code
are non-trivial. 
If they were easy, you wouldn't need a computer to do it!

There are two problems with RepeaterApp2:

\begin{itemize}
  \item It works on a two port switch, but not anything bigger.  
  And the ports absolutely have
  to be numbered 1 and 2 \ldots otherwise, the whole program doesn't work.
  And those ports need to be functioning.
  \item More subtly, this program can drop packets.  
  There is a short lag in between when the switches come up and the \texttt{self.update()} installs
  the policies. 
  During this lag, packets will arrive at the controller by default and get dropped by the 
  default \texttt{packet\_in} handler in Frenetic.   
\end{itemize}

We will correct both of these problems in section~\ref{section:stateless}

\section{Combining NetKAT Policies}
\label{section:combining}

In our Repeater2 network app, the two rules have the \texttt{Seq} operator >> in them, then the two rules 
are joined together with \texttt{Union} or |.  
So when do you use one or the other?
The following principle is easy to remember and apply.

\begin{principle}
Use \texttt{Seq} or \texttt{>>} between filters and their actions to form a rule.
Use \texttt{Union} to combine policies that DO NOT overlap.
Use \texttt{IfThenElse} to combine policies that DO overlap. 
\end{principle}

To see why this is so, let's go back to the definition of \texttt{Seq} and \texttt{Union}.
A \texttt{Seq} of policies applies each policy sequentially to each packet.
A \texttt{Union}, on the other hand, makes a duplicate of the packet and sends each copy through each policy 
simultaneously.

A \texttt{Union} after a \texttt{Filter} is useless because \texttt{Filter} does nothing but grab a subset 
of packets.
\texttt{Union} makes two copies of the packet, one which goes through the filter, and one which does another action.
But filtering, by itself, does nothing useful to the packet, and after it's over, if there are no 
further actions, the packet is dropped. 
If your complete policy was \texttt{Filter(SwitchEq(2) \& PortEq(9))}, nothing will happen to the packet, even
if it arrived on switch 2, port 9.  

Suppose you have two rules:

\begin{itemize}
  \item Filter(TcpPort(80)) >> drop Drops non-encrypted HTTP traffic 
  \item Filter(SwitchEq(1) \& PortEq(2)) >> Send(1) Sends port 2 traffic out port one.
\end{itemize}

Let's say you combine these with:

\begin{description}
  \item[Union] - all packets going to port 80 from port 2 will be copied twice.  
  The first will be dropped by the first Filter, the second will be output by the second Filter, ensuring the
  packet will go out anyway. 
  \item[Seq] - all packets going to port 80 from port 2 will be dropped by the first rule and will never
  make it to the second.   
\end{description}

TODO: Explain this better.  

\section{Keeping It Stateless}
\label{section:stateless}

\begin{principle}
When you install a new switch policy, do not assume it's there before the next matching packet arrives.
\end{principle}

\begin{principle}
Build mechanisms to automatically recreate the state if the net app dies.
\end{principle}