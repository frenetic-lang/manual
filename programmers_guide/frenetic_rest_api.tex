% !TEX root = frenetic_programmers_guide.tex

\chapter{Frenetic REST API}

The Frenetic Python language bindings are only one way to write network applications.  The Frenetic
HTTP Controller accepts NetKAT policies, configuration settings through JSON, and pushes network
events like Packet In's through JSON.  So any programming language that can speak HTTP and JSON (pretty
much all of them) can talk to Frenetic.  

\section{REST Verbs}
\label{frenetic_rest_api:urls}

The following verbs are implemented in the HTTP Controller:

\bigskip
\begin{tabularx}{\linewidth}{lXX}
\texttt{POST /$client\_id$/update\_json} & Update the network-wide NetKAT policy, compile the tables, and send the
resulting tables to the switches. \\
\texttt{GET /$client\_id$/event} & Get the latest waiting network events \\
\texttt{GET /current\_switches} & Return a list of connected switches and ports \\
\texttt{POST /pkt\_out} & Send a packet out via OpenFlows Packet Out message \\
\texttt{GET /version} & Return Frenetic version \\
\texttt{GET/config} & Get current Frenetic configuration \\
\texttt{POST /config} & Update current Frenetic configuration \\
\texttt{GET /query/$name$} & Get current statistics bucket contents \\
\texttt{GET /port\_stats/$sw$/$port$} & Get port statistics contents for switch $sw$ and port $port$ \\
\texttt{POST /$client\_id$/update} & Same as \texttt{update\_json} but the policy is in Raw NetKAT format 
\end{tabularx}

Of these verbs, you'll be calling \netkat{update_json} and \netkat{event} the most, so we'll discuss 
these in detail.  The rest are outlined in Section \ref{reference:commands}.

\section{Pushing Policies}

You push policies to Frenetic by using the \texttt{POST /$client\_id$/update\_json} verb.  The 
$client\_id$ is identical to the $client\_id$ instance variable set in our Python language examples.
To make the policy updates smaller, you can split the policy into multiple client id's, then 
call \netkat{update_json} on just that portion to update the policy.   Frenetic recompiles all the
policies together with an implicit \netkat{Union} -- so there can be no overlapping rules in each 
of the clients.  

The data you push is a JSON representation of the NetKAT policy.  For example, this policy using
the Python bindings:

\begin{minted}{python}
Filter(PortEq(1)) >> SetPort(2) |
Filter(PortEq(2)) >> SetPort(1) 
\end{minted}

Will look like this in JSON:

\begin{minted}{json}
{  "type": "union",
   "pols": [
     { "type": "seq",
       "pols": [
         { "type": "filter", 
           "pred": { 
              "type": "test", 
              "header": "port",
              "value": { "type": "physical", "port": 1 }
           }
         },
         { "type": "mod",
           "header": "port",
           "value": { "type": "physical", "port": 2 }
         }
       ]  
     },
     { "type": "seq",
       "pols": [
         { "type": "filter", 
           "pred": { 
              "type": "test", 
              "header": "port",
              "value": { "type": "physical", "port": 2 } 
           }
         },
         { "type": "mod",
           "header": "port",
           "value": { "type": "physical", "port": 1 }
         }
       ]  
     }
   ]
}
\end{minted}

Figuring out the JSON by hand is pretty laborious.  So you can make the Python REPL and Frenetic 
bindings do the work for you like this:

\begin{minted}{console}
frenetic@ubuntu-1404:~$ python
Python 2.7.6 (default, Jun 22 2015, 17:58:13)
[GCC 4.8.2] on linux2
Type "help", "copyright", "credits" or "license" for more information.
>>> import frenetic
>>> from frenetic.syntax import *
>>> pol = Filter(PortEq(1)) >> SetPort(2) | Filter(PortEq(2)) >> SetPort(1)
>>> pol.to_json()
{'pols': [{'pols': [{'pred': {'header': 'location', 'type': 'test', 'value': 
{'type': 'physical', 'port': 1}}, 'type': 'filter'}, {'header': 'location', 
'type': 'mod', 'value': {'type': 'physical', 'port': 2}}], 'type': 'seq'}, 
{'pols': [{'pred': {'header': 'location', 'type': 'test', 'value': {'type': 
'physical', 'port': 2}}, 'type': 'filter'}, {'header': 'location', 'type': 
'mod', 'value': {'type': 'physical', 'port': 1}}], 'type': 'seq'}], 'type': 
'union'}
>>>
\end{minted}

The complete catalog of JSON syntax for NetKAT predicates and policies is in Chapter 
\ref{netkat_reference}

\section{Incoming Events}

Calling the \texttt{GET /$client\_id$/event} verb gets a list of incoming network events.
The call will block until events become available, which is why the Python bindings are 
implemented asynchrononously through callbacks.   The events come back as a 
JSON array, so there could be more than one.

All incoming events are listed in Section \ref{netkat_reference:events}.  The most
prevalent one is \netkat{packet_in}, so your call to \texttt{GET /$client\_id$/event} might
return something like this:

\begin{minted}{json}
[
  { "type": "packet in", 
    "switch id": 1981745, 
    "port id": 1, 
    "payload": {
      "id": 19283745, 
      "buffer": "AB52DF57B12BF87216345" 
    }
  }
  { "type": "packet in", 
    "switch id": 923462, 
    "port id": 2, 
    "payload": {
      "buffer": "AB52DF57B12BF87216345" 
    }
  },
]
\end{minted}

The payload may be either buffered or unbuffered -- the absence of an \netkat{id} attribute 
indicates the packet is unbuffered, in which case the \netkat{buffer} attribute is the entire
contents of the packet.  (See Section \ref{intro:buffering} for an explanation of how buffering
works.)

The contents are encoded in Base 64, and most languages have decoders for this type.  You will
need to write or acquire a packet parser to extract and test data from inside the packet -- RYU's
packet library is good on the Python side. 