% !TEX root = frenetic_programmers_guide.tex

\chapter{Productionalizing}

Once you have your Frenetic-based application written and debugged on Mininet, you can run it in on a 
physical network testbed or in production.  

\section{Installing Frenetic on Bare Metal Linux}
 \label{productionalizing:install}
 
We've been running all of our experiments in Virtual Machines, and you can certainly use VM's for
production as well.  You simply need to set up port forwarding for the OpenFlow port (6633 by 
default on OpenFlow 1.0).  To do this, you can add the following to the file \codefilename{Vagrantfile}:

\inputminted[firstline=29,lastline=30]{ruby}{code/productionalizing/Vagrantfile} 

In production, however, the speed of delivering Packet In's, Packet Out's, and rules to switches is crucial so
removing the extra VM network stack overhead makes sense.

To install Frenetic and your network application on a bare metal server, first install Ubuntu Server 14.04.  
This is the OS used by Frenetic VM.  

Next install Git from your administrative login:

\begin{minted}{console}
ubuntu@ubuntu:~$ sudo apt-get update
ubuntu@ubuntu:~$ sudo apt-get install git
\end{console}

Retrieve the code from the Frenetic-vm Github repository:

\begin{minted}{console}
ubuntu@ubuntu:~$ git clone https://github.com/frenetic-lang/frenetic-vm
\end{console}

Then run the administrative provisioning script.  This installs all the supporting software for 
Frenetic, and should take somewhere from 5 to 30 minutes.

\begin{minted}{console}
ubuntu@ubuntu:~$ cd frenetic-vm
ubuntu@ubuntu:~/frenetic-vm$ sudo ./root-bootstrap.sh 
\end{console}

Create a non-root user under which Frenetic and the application will run.  In our examples, we will
use \codefilename{frenetic} user:

\begin{minted}{console}
ubuntu@ubuntu:~$ sudo adduser frenetic
Adding user `frenetic' ...
Adding new group `frenetic' (1001) ...
Adding new user `frenetic' (1001) with group `frenetic' ...
Creating home directory `/home/frenetic' ...
Copying files from `/etc/skel' ...
Enter new UNIX password:
Retype new UNIX password:
passwd: password updated successfully
Changing the user information for frenetic
Enter the new value, or press ENTER for the default
	Full Name []:
	Room Number []:
	Work Phone []:
	Home Phone []:
	Other []:
Is the information correct? [Y/n]
\end{minted}

Add the \codefilename{frenetic} user to do the \codefilename{/etc/sudoers} file so it can perform administrative
tasks.  This is required for the next scripts to run, and if you wish, you can revoke sudo permissions
afterwards (they are not required to run Frenetic or the network application):

\begin{minted}{console}
ubuntu@ubuntu:~$ sudo adduser frenetic sudo
Adding user `frenetic' to group `sudo' ...
Adding user frenetic to group sudo
Done.
\end{minted}

Copy the \codefilename{user-bootstrap.sh} script to the \codefilename{frenetic} user's workspace:

\begin{minted}{console}
ubuntu@ubuntu:~/frenetic-vm$ sftp frenetic@localhost
The authenticity of host 'localhost (127.0.1.1)' can't be established.
ECDSA key fingerprint is 3c:f4:ba:46:f0:b7:1f:74:cd:34:33:f9:12:5e:3a:3e.
Are you sure you want to continue connecting (yes/no)? yes
Warning: Permanently added 'localhost' (ECDSA) to the list of known hosts.
frenetic@localhost's password:
Connected to localhost.
sftp> put user-bootstrap.sh
Uploading user-bootstrap.sh to /home/frenetic/user-bootstrap.sh
user-bootstrap.sh                                                                                 100%  296     0.3KB/s   00:00
sftp> quit
\end{minted}

Then switch over to the \codefilename{frenetic} workspace and run the script.  This installs and 
compiles the latest version of Frenetic, and takes about 15 minutes.

\begin{minted}{console}
ubuntu@ubuntu:~$ sudo -i -u frenetic
frenetic@ubuntu:~$ mkdir src
frenetic@ubuntu:~$ chmod u+x user-bootstrap.sh
frenetic@ubuntu:~$ ./user-bootstrap.sh
\end{minted}

Lastly, you need to retrieve your network application code and store it in the \codefilename{frenetic}
user's workspace.  We generally store network applications
in Github and just use Git tools to retrieve the latest Master copy.  That way it's easy to 
update later.  In the examples below, we'll assume the network application is in 
\codefilename{src/l2_learning_switch/learning4.py}.  

\section{Control Scripts}

Ubuntu 14.04 includes the Upstart utility to wrap programs as daemons.  This is handy for 
starting and stopping both Frenetic and your network application.  Creating Upstart scripts
for both handles logging, and automatically starting up the daemons from a cold boot.

Here is a sample Upstart script for Frenetic, which you can place in 
\codefilename{/etc/init/frenetic.conf} (you must do this with \codefilename{sudo}):

\inputminted{bash}{code/productionalizing/frenetic.conf} 

And here is one for your network application.  We'll assume the network application is in 
\codefilename{src/l2_learning_switch/learning4.py}.  

\inputminted{bash}{code/productionalizing/l2_learning_switch.conf} 

And you can control both processes with standard Ubuntu commands:

\begin{description}
\item[sudo service frenetic start] Starts the Frenetic controller.  Note, only one copy can be running, so
issuing it when Frenetic is already running does nothing.   
\item[sudo service frenetic stop] Stops the Frenetic controller.
\item[sudo service frenetic status] Displays the status of Frenetic - whether it's running or not.
\end{description}

The commands are similar for the network application \codefilename{l2_learning_switch}.  

One nice feature of Upstart is logging.  Using the Upstart script for Frenetic as above places the logs 
in \codefilename{/var/log/upstart/frenetic.log}.  By default logs are rotated daily, compresssed, and 
deleted after 7 days.  

