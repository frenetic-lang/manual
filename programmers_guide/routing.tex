\chapter{Routing}

\section{Design}
\label{routing:design}

Every OpenFlow enabled device is called a \emph{switch}, but you should not confuse it
with a traditional L2 switch.  An OpenFlow switch can model just about any network device
including firewalls, load balancers, and -- as we'll see in this chapter -- routers.  

In Chapter \ref{multitswitch_topologies}, we saw two ways of modelling the network topology:
statically with a \netkat{.dot} file and dynamically via Frenetic.  One advantage of a
static topology is you can share it between Mininet and your application.  That 
way you can model more difficult topologies completely, and only change one file
to change the design.  

Until now we've been using Mininet's predefined topologies: simple and tree.  We
can define a custom Mininet toplogy by writing some Python code calling the Mininet
API's.  

\section{Configuring Route Tables}

\section{Using Link State}

