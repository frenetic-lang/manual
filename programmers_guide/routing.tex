% !TEX root = frenetic_programmers_guide.tex

\chapter{Routing}

\section{Design}
\label{routing:design}

Up until now, we've been dealing with OSI Layer 2 technologies -- those that operate at the
Ethernet packet level.  Now we'll step one layer up the stack to Layer 3: Internet Protocol
or IP.  

From the IP perspective, the Internet is just a bunch of LAN's organized into networks, 
then further divided into subnets.  For our purposes, we'll concentrate on
IP Version 4 or IPv4, which is currently the most popular of the two IP versions (the other being
IP Version 6 or IPv6).  

A particular IPv4 address, for example 10.0.1.3, may be part of a subnet
with other hosts.  We may set up a subnet labelled 10.0.1.0/24, which means the first 24 bits
comprise the subnet number, and the last $32 - 24 = 8$ bits are the host number.  In our example, the
subnet number is 10.0.1 and the host is 3.  Because the host number is 8 bits, this means our example
host lives in a subnet with up to 256 neighbors.  Some neighbors are reserved:

\begin{description}
\item[Host 0] in this case 10.0.1.0, usually has no host assigned to it. 
\item[Host 1] in this case 10.0.1.1, is usually the default gateway of the subnet, which we'll see in 
a minute.
\item[Host $n$ - 1] in this case 10.0.1.255, is the subnet broadcast address.
\end{description}

This leaves you with $n$ - 3 ``real'' neighbor hosts.  You can think of a subnet as a gated community
where the most common action is to talk with your neighbors, but not with those outside your
subnet.  To do the latter, you need an intermediary \ldots in IP, that intermediary is a router.

So suppose we have two subnets, each with two hosts, organized like this:

\includegraphics{routing_topo.pdf}

There is no such thing as speaking IP \emph{natively} over a network.  You must speak the wire protocol,
which in most cases is Ethernet.  Because IP applications like your browser and Ping can only speak
in IP addresses, there must be a translation mechanism between IP addresses and Ethernet addresses.
That translation mechanism is Address Resolution Protocol or ARP.  

A typical conversation between two hosts on the same subnet, looks something like this:

\begin{enumerate}
\item Host 10.0.1.2 wants to communicate with 10.0.1.3, but doesn't know it's MAC address.
\item It broadcasts over Ethernet an ARP request, "Who has 10.0.1.3" ?
\item Assuming 10.0.1.3 is up, it sends back an ARP reply, "10.0.1.3 is at 11:22:33:44:55:66", which is
presumably its own MAC address.
\item Host 10.0.1.2 then constructs an Ethernet packet with destination 11:22:33:44:55:66 and sends it off.
\end{enumerate}

We haven't had to think about ARP in previous chapters, because the hosts have seamlessly handled it
for us.  Our L2 learning switch handles Ethernet broadcasts and Ethernet unicasts just fine, so every
part of the conversation above was handled by it seamlessly.

Now, throw a router into the mix and the conversation gets slightly more complicated.  Host IP
stacks now distinguish between hosts on their own subnet and hosts on other subnets.  If
10.0.1.2 wants to talk to 10.0.2.3, the host won't just send out an ARP request for 10.0.2.3 -- it's
on another network.  Hosts can have routing tables to tell it where to direct inter-network
traffic, but in most cases the routing table contains one entry: the default gateway.  The default
gateway is usually set by DHCP, but it's generally a special IP address on the subnet where the
router lives. In this case the conversation becomes:

\begin{enumerate}
\item Host 10.0.1.2 wants to communicate with 10.0.2.3, but doesn't know it's MAC address.  Since
10.0.2.3 is on different subnet, and there's no routing table entry for a subnet including 10.0.2.3,
it decides to send to a default gateway 10.0.1.1
\item It broadcasts over Ethernet an ARP request, ``Who has 10.0.1.1?''
\item The router sends back an ARP reply, "10.0.1.1 is at 11:00:00:00:00:00", which is
the MAC address of the router port hooked into subnet 10.0.1.0/24.
\item Host 10.0.1.1 then constructs an Ethernet packet with destination 11:00:00:00:00:00 and sends it off.
\end{enumerate}

And now the router can do its thing:

\begin{enumerate}
\item The destination IP is 10.0.2.3, and the router knows it has subnet 10.0.2.0/24
on port 2.  But it doesn't know the MAC address of 10.0.2.3.
\item The router broadcasts an ARP request over port 2 ``Who has 10.0.2.3?''
\item That host responds with an ARP reply ``10.0.2.3 is at ff:ee:dd:cc:bb:aa'', which is its 
own MAC address.
\item Router constructs an Ethernet packet with the router MAC address connected to that
subnet as its source,
and ff:ee:dd:cc:bb:aa as its destination and sends it off.
\end{enumerate}

A couple of things to notice here:

\begin{itemize}
\item Only IP traffic gets routed.  
\item Only Ethernet sources and destinations are changed in the packet.  The IP addresses
stay the same no matter where they are in the process.
\item A router must buffer packets until the ARP replies return.  For subsequent requests, it caches the
IP-to-MAC translation, like a learning switch does with MACs.
\item If the router receives a packet bound for a network not directly connected to it, the router
itself has a default gateway it can send to.  
\end{itemize}

So the router does three basic things: answers ARP requests, sends ARP requests, and routes
a packet to the ``next hop''.  Of course real routers do much more than that: they maintain routing tables,
translate network protocols, drop blatantly malicious traffic, and so on.  But we'll concentrate 
on the three core router functions here.

\section{Modeling The Topology}
\label{routing:topo}

Every OpenFlow enabled device is called a \emph{switch}, but you should not confuse it
with a traditional L2 switch.  An OpenFlow switch can model just about any network device
including firewalls, load balancers, and -- as we'll see in this chapter -- routers.  

In Chapter \ref{multitswitch_topologies}, we saw two ways of modeling the network topology:
statically with a \netkat{.dot} file and dynamically via Frenetic.  One advantage of a
static topology is you can share it between Mininet and your application.  That 
way you can model more difficult topologies completely, and only change one file
to change the design.  

So here is our DOT file, from \codefilename{routing/topology.dot}:

\inputminted{python}{code/routing/topology.dot}

We've added a few more attributes to accommodate IP.  In particular:

\begin{description}
\item[router] is set to true on the device acting as the router
\item[ip] addresses are assigned to each host 
\item[gateway] addresses is the default gateway to which the host sends packets.  All packets not
bound for hosts in the same subnet go here.   
\end{description}

Here's the GraphViz generated diagram from the above:

\includegraphics{routing_topology.pdf}

Until now we've been using Mininet's predefined topologies: simple and tree.  We
can define a custom Mininet toplogy by writing some Python code calling the Mininet
API's.  

The following code is in \codefilename{routing/mn_dot_topology.py}:

\inputminted{python}{code/routing/mn_dot_topology.py}

This program relies on the \python{agraph} API, which we used in Chapter \ref{multiswitch}, and 
the Mininet API, documented at: http://mininet.org/api/annotated.html.  The interesting calls are:

\begin{description}
\item[addSwitch()] which adds a new Mininet switch with a particular DPID.
\item[addHost()] which adds a host
\item[addLink()] which adds virtual ``wire'' between a host and a switch, or two switches.
\end{description}

In this program, we're careful to set the port ids to some defaults.  Although that's not
technically necessary, since L2 learning will establish a table of MAC-to-port mappings,
it makes debugging easier.    

To run this Mininet topology, you simply run this python file as root:

\begin{minted}{console}
vagrant@frenetic:~/manual/programmers_guide/code/routing$ sudo python mn_dot_topology.py
*** Removing excess controllers/ofprotocols/ofdatapaths/pings/noxes

  ... a lot of cleanup messages 

*** Cleanup complete.
Unable to contact the remote controller at 127.0.0.1:6633
Network ready
Press Ctrl-d or type exit to quit
mininet>
\end{minted} 

Finally, we set up a static routing table.   Subnets directly connected to a router are usually 
configured statically in this manner, while indirectly-connected subnet entries are handled by
a route advertising protocol like OSPF.  Our test network won't have the latter.

We use simple JSON to define the subnets in \codefilename{routing/routing_table.json}:

\inputminted{json}{code/routing/routing_table.json}

This information isn't needed to \emph{build} the network, but it's crucial for its operation.

\section{A Better Router}

One of the problems with traditional routing is the ARP process.  The router must have gobs of memory 
to buffer packets waiting for ARP replies \ldots and those replies may never arrive!  But here, SDN 
can help us build a faster router.

In Chapter \ref{multiswitch_topologies} we saw how a global network view helps us build a loop-free
network without the hassle of spanning tree.  We can use this same global network view to build
routing without relying as much on ARP.  

The key is in the L2 learning tables.  If 10.0.1.2 wants to communicate with 10.0.2.3, we probably know
both of their MAC addresses.  10.0.1.2 cannot simply address its packet directly to that MAC because
it's not on the same switch (and Ethernet, by itself, is not routable).  But the router can use 
10.0.2.3's MAC address to perform the second hop.  It can skip the buffering and ARP request step
altogether.

But what if the MAC address of 10.0.2.3 is \emph{not} known?  
This looks like the case in the L2 learning switch when we simply ``punt'' and flood the packet
over all ports.  
Unfortunately the router can't do that with L3 packets.  If you simply stamp an Ethernet broadcast
address in the same packet as the destination IP, the end hosts will see the packet, note that 
the IP and MAC addresses don't match, and drop the packet.
In this case, the router \emph{does} need
to enqueue packets, send the ARP request, monitor the response, and release the packets when 
their destination MAC is known.   

Under these conditions, the L2 switches need very few modifications.  When a host needs to send an 
internetwork packet, it always starts by sending the packet to the default gateway.  From that 
perspective, the default gateway is no different than any other host.  If the sender knows 
its MAC address, it simply sends it.  If it doesn't, it broadcasts an ARP request for it and the
router responds.  

The router needs to implement a rule for each learned MAC in the network.  For each
$mac$ and address $ip$, you first determine the router port $rp$ and the MAC address of that
port $rtrmac$.  That can be calculated since
we know which subnet is connected to each router port.  Then we write the rule:

\begin{minted}{python}
Filter(EthTypeEq(0x800) & IP4DstEq(ip)) >> \
  SetEthSrc(rtrmac) >> SetEthDst(mac) >> SetPort(rp) 
\end{minted}

Two more things: we need to catch all remaining unlearned IP destinations, and 
we need to catch all ARP requests so that we can answer requests for the default gateway:

\begin{minted}{python}
Filter(EthTypeEq(0x800, 0x806) ) >> SendToController("router") 
\end{minted}

Note that we don't need to \emph{answer} all ARP requests, necessarily \ldots and in fact we'll see
ARP requests for intra-network hosts because they are broadcast over the subnet.  We can ignore those
since they'll be answered by the host itself.  The only ones we reply to are those for the default 
gateway.  

\section{Modularization}

When we get done, our network application will perform the job of both the switches and the router.  
Up until now, we have written our applications as one subclass of \python{Frenetic.App}, but here
it makes sense to modularize the application into a switch part and a router part.  We call these
handlers, and they implement the same method signatures that Frenetic applications do.  
They are not subclasses of \python{Frenetic.App} -- making them subclasses would give them each 
their own event loops and asynchronous calls, which simply makes things too complicated.   However, they
could be turned into their own freestanding Frenetic applications simply by making them subclasses.

The main application looks like this, from \codefilename{routing/routing1.py}:

\inputminted{python}{code/routing/routing1.py}

Notice how it creates one NIB, and passes these to both switch and router handlers.  This allows
them to share state.  But learning a new MAC on the switch should trigger a policy recalculation on
the router.  Rather than coding this dependency into the router (which then couples it to the switch 
handler), we handle the recalculation here.  

The recalculation uses a trick from the Tornado Python library.  If you simply call the 
\python{update_and_clear_dirty()} function, many successive packets may cause the updates to 
happen in a random order since the requests are handled asynchronously.  This can cause older
calculated rules sets to overwrite newer ones.  By using \python{IOLoop.add_timeout()}, we enqueue
all the recalculations so they occur in order.  

There's not a lot of code in this app -- most of the actual work is delegated to the handlers
\python{SwitchHandler} and \python{RouterHandler}.  Each handler does two main tasks: (a) review
incoming packets and (b) contribute their portion of the network-wide policy based on the NIB.  
Since the switches and router policies are non-overlapping (they have different \netkat{SwitchEq} filters)
simply \netkat{Union}'ing them together gives you the network-wide policy.  
In this app, all the handler workflow is hard-coded, but you can imagine a more dynamic 
main program that would register handlers and dynamically delegate events based on signatures. 
That's overkill for our routing application.

The NIB, in \codefilename{routing/network_information_base.py} looks a lot like the NIB
we use in multiswitch handling.  IP information, though, makes the tables a bit wider, 
so we move to using objects instead of tuples.  These two classes model devices (connected hosts
and gateways) and subnets.    

\inputminted[firstline=5,lastline=31]{python}{code/routing/network_information_base.py} 

The hosts table is now a dictionary of MAC addresses to \python{ConnectedDevice} instances. 

The initialization procedure reads the fixed configuration from the Routing Table and topology
files.  It follows the same general outline as the Mininet custom configurator.

\inputminted[firstline=56,lastline=87]{python}{code/routing/network_information_base.py} 

The learning procedure adds the IP field, which may be passed in as \python{None}
for non-IP packets.  The first packet from a device might very well be non-IP, as in a DHCP 
request (even though DHCP asks for an IP address, the DHCP packet itself is not an IP packet).
That might trigger learning on the L2 switch, which records its port and MAC, but no IP address.  
The router can later call this procedure to fill in the missing IP.  

The NIB now has \python{dirty} flag
which learning/unlearning a MAC can set or.  The main handler uses this to determine whether to do 
a wholesale recalculation of the switch and router policies.

\inputminted[firstline=186]{python}{code/routing/network_information_base.py} 

The switch handler is virtually identical to the switching application of Chapter \ref{multiswitch}.  
The main difference is extra handling in the MAC learning procedure -- if the packet is an
IP packet, we extract the extra IP address and learn that as well.  One important except is the 
\emph{internal port}, meaning any ports connected to other switches or routers.  Because the 
routing process changes the Ethernet Source MAC, we can't count on it to match the Source IP address,
which itself never changes.  So though we learn the MAC and port of an internal port, we ignore the
IP address of any packets arriving there.  

This code is from \codefilename{routing/switch_handler.py}:

\inputminted[firstline=33]{python}{code/routing/switch_handler.py}

The interesting processing happens in \codefilename{routing/router_handler.py}. First, we set up 
objects to hold queued packets waiting for an ARP reply:

\inputminted[firstline=8,lastline=12]{python}{code/routing/router_handler.py} 

Policies are constructed from the learned MAC table, similarly to the switches.  The big differences
are we look at the entire network-wide MAC table, and we look only at those entries with IP 
addresses.

\inputminted[firstline=24,lastline=53]{python}{code/routing/router_handler.py} 

Here you can see the learned MAC policies, the catch-all subnet policies, and the ARP policy are 
installed.  ARP replies are constructed from scratch using the Frenetic Packet object described in
Section \ref{introduction:packet_in}:

\inputminted[firstline=94,lastline=110]{python}{code/routing/router_handler.py} 

The Packet In handler deals primarily with ARP requests and replies:

\inputminted[firstline=111,lastline=147]{python}{code/routing/router_handler.py} 

And with IP packets:

\inputminted[firstline=149]{python}{code/routing/router_handler.py} 

The queuing and dequeuing procedures are straightforward.  Note that a huge stream of packets to a 
disconnected IP address could easily overwhelm this implementation, and so caps on the queues should
probably be enforced:

\inputminted[firstline=54,lastline=92]{python}{code/routing/router_handler.py} 

\section{Summary}

The IP routing process need a summary.  

In the next chapters, we build on layer three technology.  In chapter \ref{} we build Network Address Translation
which is like routing but changes the IP address and ports as well.  From there it's a small step to 
Firewall and Load Balance technology.  
