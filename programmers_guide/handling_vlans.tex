% !TEX root = frenetic_programmers_guide.tex

\chapter{Handling VLANs}
\label{chapter:handling_vlans}

\section{VLAN Uses}
\label{handling_vlans:uses}

Back in the late 1990's when LAN switching became more prevalent, network architects ran into balancing problems.
Enterprise divisions and departments wanted to keep their 
LANs separate for security and speed reasons.  For example, departments that generated heavy traffic 
during certain times of the month (like the Accounting department) needed to be segregated from others
to prevent intolerable slowdowns in other departments.  

The problem was network boundaries had to match physical switch boundaries.  If you bought switches with
48 ports, and a department had 12 hosts, then 36 would remain unused.  Worse, if a department grew from 
48 to to 49 ports, you had to make an emergency network buy.  

Networks often solved these problems by adding routers, splitting department subnets and letting the 
routers handle traffic between them.  But routing can be expensive and slow, especially for large 
bursts of traffic.  Routing makes sense between networks you want to keep segregated, but it's
overkill between two hosts that you know you want to keep together.

Virtual LAN's, called VLAN's, emerged to solve these problems.  You can think of a VLAN as a logical 
network, mappable onto physical switches.  A VLAN is like an elastic switch -- you assign real ports to 
a particular VLAN, then can add or remove them at will.  

VLANs can span multiple switches.  So if you needed that 49th port for a department, you could assign 
an unused port on a different switch to that VLAN, and the host on it would act as if it were connected to 
the same switch.  Of course, there's technically more to it than that, especially since you need to get the
packet from one switch to the other.  But as a network operator, VLANs freed you from worrying about
those details.  

OpenFlow-based switches and Frenetic can integrate with standard VLAN technology easily.  In this 
chapter, we'll first simulate a simple VLAN in software.  Then we'll use standard 802.1Q VLANs to 
make a more flexible setup.  Then we'll use VLAN technology between switches.  

\section{A Simple, Fixed VLAN}

You've probably noticed that, as we're designing networking applications we start with something 
static and hard-coded, then gradually move to something configurable and dynamic.  This kind of 
organic growth makes network application design more agile.
It's easier to start with a simple net app, debug it, then add and test more functionality a bit at a time.  

So we'll start with a simple rule-based VLAN structure:

\begin{itemize}
\item Odd-numbered ports (1, 3, 5, \ldots) will get mapped to VLAN 1001
\item Even-numbered ports (2, 4, 6, \ldots) will get mapped to VLAN 1002
\end{itemize}

The VLAN ids 1001 and 1002 will only be used internally for this first application.  Later, we'll use
them as actual 802.1Q VLAN ids, since they fit into the range 1..4096.  

Hosts in each VLAN will talk only to hosts in their own VLAN, never to hosts in the other VLAN.  We'll talk
about how to connect the two VLANs together a little later.  

As we saw in Chapter 4, in NetKAT the L2 switch basic rule looks like this:

\begin{minted}{python}
Filter(EthDstEq("11:11:11:11:11:11")) >> SetPort(2)
\end{minted}

In our new VLAN-based network, port 2 is on VLAN 1002, to which all even-numbered ports belong.  
To segregate the networks, we'll simply extend the rule to look like this:

\begin{minted}{python}
Filter(PortEq(2,4,6,8,10)) >> Filter(EthDstEq("11:11:11:11:11:11")) >> SetPort(2)
\end{minted}

Where we cobble together $2, 4, 6\ldots$ from the even-numbered ports we know about on the switch.  

If the destination is unknown, instead of flooding packets out \emph{all} ports of the switch, 
we'll flood over ports of the switch \emph{in the same VLAN}.  In other words, the Packet Out action 
for a packet arriving at port 4 will be:

\begin{minted}{python}
SetPort(2,6,8,10)
\end{minted}

We'll add some methods onto the \python{NetworkInformationBase} object to do some VLAN mapping.  The following
code is in \codefilename{code/handling_vlans/network_information_base_static.py}:

\inputminted[firstline=60]{python}{code/handling_vlans/network_information_base_static.py}

The \python{policy_for_dest} method gets extended with the new filtering.  
The following
code is in \codefilename{code/handling_vlans/vlan1.py}:

\inputminted[firstline=23,lastline=30]{python}{code/handling_vlans/vlan1.py}

And the \python{packet_in} handler does some extra VLAN handling.  Note that if the destination port is already
learned, we verify it's connected to the same VLAN as the source.  That way, hosts cannot just forge 
destination MAC addresses in the other VLAN and subvert security.  

\inputminted[firstline=43,lastline=73]{python}{code/handling_vlans/vlan1.py}

Otherwise, learning switch internals stay the same.  Using a \python{single,4} topology in Mininet,
a pingall:

\begin{minted}{console}
mininet> pingall
*** Ping: testing ping reachability
h1 -> X h3 X
h2 -> X X h4
h3 -> h1 X X
h4 -> X h2 X
*** Results: 66\% dropped (4/12 received)
mininet>
\end{minted}

reveals that only odd-numbered ports talk to odd-numbered ports, and only even-numbered ports talk to
even-numbered ports.  We now have two segregated VLANs.

\section{Standard, Dynamic VLANs}

So now we want to make our setup a little more flexible.  Our program effectively codes the port-to-VLAN
mappings.  We \emph{could} also store this mapping in a configuration file or a database.  But even that's
too static.  We want network operators to be able to change VLAN-to-port mappings on the fly, simply
by changing the interface configuration on the switch.

So how are VLANs assigned to ports?  The procedure varies from switch to switch, but in most cases you
create an \emph{untagged} interface.  On an HP switch, for example, you configure the interface information
like this:

\begin{minted}{console}
coscintest-sw-ithaca# config
coscintest-sw-ithaca(vlan-1)# vlan 1001
coscintest-sw-ithaca(vlan-1001)# untagged A1
\end{minted}

This assigns the port A1 to VLAN 1001.  
That means any packets going into this port without any VLAN
information (i.e. untagged) are implicitly tagged with a fixed VLAN id.   This VLAN id is visible in NetKAT
predicates, and in the \python{packet_in} hook, even though they technically don't belong to the 
packet that entered the switch.  (This is a rare instance where packet modifications are applied
before Frenetic gets ahold of them.)

Outgoing packets on these ports have VLAN information stripped from them.  That's because we generally 
want hosts to be unencumbered by VLAN information.  This allows us to move the host from one VLAN to another
without reconfiguring the host itself.  

Unfortunately on OpenVSwitch, which Mininet uses 
underneath the covers, configuration is not so easy.  When OpenVSwitch is running in 
OpenFlow mode, you cannot assign VLAN tags to ports.  So we're going to use a Mininet
custom topology, which is a Python program that sets up our toy network.  In the 
following case, it'll look almost exactly like a single-switch 4 node network, except the 
VLAN tags will be added on the host itself.  Though this is not exactly what happens 
in real life, we can at least use the same net app for development and production work.

\inputminted{python}{code/handling_vlans/mn_custom_topo.py}

We start up this custom Topology like so, which lands at a Mininet prompt just as it 
does for canned topologies:

\begin{minted}{console}
frenetic@ubuntu-1404:~/manual/programmers_guide/code/handling_vlans$ sudo python mn_custom_topo.py
Unable to contact the remote controller at 127.0.0.1:6633
Network ready
Press Ctrl-d or type exit to quit
mininet> net
h1 h1-eth0.1001:s1-eth1
h2 h2-eth0.1002:s1-eth2
h3 h3-eth0.1001:s1-eth3
h4 h4-eth0.1002:s1-eth4
s1 lo:  s1-eth1:h1-eth0.1001 s1-eth2:h2-eth0.1002 s1-eth3:h3-eth0.1001 s1-eth4:h4-eth0.1002
c1
mininet>
\end{minted}

In our net app, learning VLAN-to-port mappings is very similar to learning MAC-to-port mappings.  In fact, 
we can do it all at the
same time.  We assume that an incoming packet for port $p$ with a VLAN tag of $v$ can be learned as a 
VLAN-to-port mapping.  If a port is ever reassigned to another VLAN, the same \python{port_down} and
\python{port_up} hooks will be fired as for MAC moves, so we can unlearn the VLAN mappings at the same
time.  

We'll store the VLAN-to-port mappings in a Python dictionary, similar to our MAC-to-port mappings.  
One difference is the values of this dictionary will be lists, since a VLAN (the key of our dictionary)
will be assigned to many ports.  This code will will replace our static VLAN view in 
\codefilename{code/handling_vlans/network_information_base_static.py}:

\inputminted[firstline=65]{python}{code/handling_vlans/network_information_base_dynamic.py}

Then we tweak the \python{learn} method to learn both MACs and VLANs:

\inputminted[firstline=13,lastline=24]{python}{code/handling_vlans/network_information_base_dynamic.py}

And in the \python{delete_port} method, we clean up any lingering VLAN-to-port mappings for that port:

\inputminted[firstline=55,lastline=60]{python}{code/handling_vlans/network_information_base_dynamic.py}

Since we now have packets tagged with VLANs, the NetKAT policies no longer need to reference list of ports.
They will change to look like:

\begin{minted}{python}
Filter(VlanEq(1002)) >> Filter(EthDstEq("11:11:11:11:11:11")) >> SetPort(2)
\end{minted}

Which we enshrine in the \python{policy_for_dest} method, listed in 
\codefilename{code/handling_vlans/vlan2.py}:

\inputminted[firstline=23,lastline=29]{python}{code/handling_vlans/vlan2.py}

In the \python{packet_in} handler, we read the VLAN from the packet instead of precomputing it:

\inputminted[firstline=42,lastline=71]{python}{code/handling_vlans/vlan2.py}

The first time we run the app, the performance seems abysmal:

\begin{minted}{console}
mininet> pingall
*** Ping: testing ping reachability
h1 -> X X X
h2 -> X X X
h3 -> h1 X X
h4 -> X h2 X
*** Results: 83% dropped (2/12 received)
\end{minted}

but this is normal.  When h1 starts pinging, there are no other hosts assigned to its VLAN.  
Same with h2.  When h3 starts pinging, finally the program can send the pings over to the only
host it knows on the same VLAN - h1.  h4 is similar.  The important thing is that, at no time do
hosts on different VLANs talk to one another.  

And now our switch handles VLANs dynamically.  You can create new VLANs, assign ports to them, rearrange or
dismantle them altogether, all without restarting the network application.  

\section{Summary}

Virtual LANs, or VLANs, introduce a level of indirection in creating a network.  Before them, physical
switches forced physical boundaries on the design.  Overprovisioning meant that many switches had
unused ports while other switches were filled to the brim.  

In this chapter we created:

\begin{itemize}
\item An application that simulates VLANs algorithmically
\item And an extended, dynamic application that reads VLAN information from packets and adjust accordingly, 
similar to our learning switch.
\end{itemize}

Note that you can combine these two approaches.  You can enforce certain rules, such as "ports 1-35 will always
be access ports assigned to VLANs 1000-1999, ports 35 and up will be on the management VLAN 1."  This is an
excellent use case for SDN -- your rules will never be hampered by an inflexible switch.

Up until this point, we have been working with one switch, which makes things easy to design.  But multiple
switches will extend the range of our network greatly.  They come with their own sets of challenges though,
which we'll work through in our next chapter.  