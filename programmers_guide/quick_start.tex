% !TEX root = frenetic_programmers_guide.tex

\chapter{Quick Start}

In this book, you will use the open source software Frenetic to create a fully programmable network.  
For the moment, let's assume you're familiar with Software Defined Networking and the OpenFlow protocol, 
and just dive right in.  (If you're not, don't worry!  
We'll introduce some bedrock concepts in the next chapter and explain everything that happened here.)  

\section{Installation}

There are several ways to get started with Frenetic, but the easiest is to use Frenetic VM.  
Frenetic itself only runs on Linux, but the Frenetic VM will run on any host system that supports 
VirtualBox, including Windows, Mac OS X and practically any version of Linux.   
Keeping Frenetic it in its own VM will keep your own system clean and neat.  
Later on, if you want to install Frenetic on a bare metal Ubuntu Linux server or network device, 
you can use the instructions in \ref{productionalizing:install}. 

First you'll need the following prerequisites: 

\begin{enumerate}
\item Install VirtualBox from \texttt{https://www.virtualbox.org/wiki/Downloads}.  Use the latest version platform package appropriate for your system.  
\item From 
\texttt{http://download.frenetic-lang.org/uservm/frenetic-uservm-current}
download the latest Frentic User VM -- this file is about 1.5 GB and
may take about 10 minutes or so to download.  
\item Unzip the file and import the \codefilename{.ova} file into VirtualBox.  This takes two minutes or
so.  
\end{enumerate}

Then you start up the Frenetic User VM from VirtualBox.  This automtically logs in to a user named 
\texttt{frenetic}.  (The password to this account is \texttt{frenetic} as well, just in case you 
need it.)

\section{What Do You Get With Frenetic VM?}

At the end of the process you will have a working copy of Frenetic with lots of useful open source infrastructure:

\begin{description}
\item[Mininet] software simulates a test network inside of Linux.  
It can model a topology with many switches and hosts.  
Writing a network application and throwing it into production is \ldots well, pretty risky, but running it on Mininet first can be a good test for how it works beforehand.  
\item[Wireshark] captures and analyzes network traffic.  
It's a great debugging tool, and very useful for sifting through piles of network packets.
\item[Frenetic].  This layer provides an easy-to-use programmable layer on top of ODL.  Its main job is to shuttle OpenFlow messages between ODL and your application, and to translate the language NetKAT into OpenFlow flow tables.  We'll see the differences between the two as we go.
\end{description}

Hmmm, that's a lot of software - what do {\it you} bring to the table?  You write your network application in Python, using the Frenetic framework.  As you'll see, it's quite easy to build a network device from scratch, and easy to grow it organically to fit your requirements.  Python is fairly popular, and knowing it will give you  a head start into Frenetic programming.  But if you're a Python novice that's OK.  As long as you know one object-oriented language fairly well, you should be able to follow the concepts.  We'll introduce you to useful Python features, like list comprehensions, as we go.  

\section{An Attempt at Hello World}

So let's dive right in.  We'll set up a Mininet network with one switch and two hosts.  
First start up Frenetic User VM from VirtualBox if you haven't already.  
Then start up a Terminal console -- two are provided in the VM under Accessories: Byobu Terminal
(which integrates nicely with tmux)
and LXTerminal (which has graphical tabs).  Either one will do.  

Let's start up a Mininet network with one switch and two nodes.

\begin{minted}{console}
frenetic@ubuntu-1404:~$ sudo mn --topo=single,2 --controller=remote
*** Creating network
*** Adding controller
Unable to contact the remote controller at 127.0.0.1:6633
*** Adding hosts:
h1 h2
*** Adding switches:
s1
*** Adding links:
(h1, s1) (h2, s1)
*** Configuring hosts
h1 h2
*** Starting controller
c0
*** Starting 1 switches
s1 ...
*** Starting CLI:
mininet>
\end{minted}

The topology Mininet constructs looks like this:

\includegraphics{simple_network}

The prompt changes to {\tt mininet>} to show your working in Mininet.  
The error message {\tt Unable to contact controller at 127.0.0.1:6633} looks a little ominous, but it's not fatal.  

You now have an experimental network with two hosts named h1 and h2.  
To see if there's connectivity between them, use the command \netkat{h1 ping h2} which means 
``On host h1, ping the host h2.''

\begin{minted}{console}
mininet> h1 ping h2
PING 10.0.0.2 (10.0.0.2) 56(84) bytes of data.
From 10.0.0.1 icmp_seq=1 Destination Host Unreachable
From 10.0.0.1 icmp_seq=2 Destination Host Unreachable
From 10.0.0.1 icmp_seq=3 Destination Host Unreachable
^C
--- 10.0.0.2 ping statistics ---
6 packets transmitted, 0 received, +3 errors, 100\% packet loss, time 5014ms
pipe 3
\end{minted}

The ping gets executed over and over again, but the \netkat{Destination Host Unreachable} message
shows it's clearly not working.  So we press CTRL-C to stop and 
quit out of Mininet:

\begin{minted}{console}
mininet> quit
\end{minted}

So by default, hosts can't talk over the network to each other.  We're going to fix that by writing a {\it network
application}.    Frenetic will act as the controller on the network, and the network application tells 
Frenetic how to behave.

\section{A Repeater}

You will write your network application in Python, using the Frenetic framework.  
Mininet is currently running in our VM under its own terminal window, and we can leave it like that.   
We'll do our programming in another window, so start up another Terminal window and
get the sample code from the Frenetic Github repository:

\begin{minted}{console}
frenetic@ubuntu-1404:~$ git clone https://github.com/frenetic-lang/manual
Cloning into 'manual'...
remote: Counting objects: 102, done.
remote: Total 102 (delta 0), reused 0 (delta 0), pack-reused 102
Receiving objects: 100% (102/102), 1.46 MiB | 728.00 KiB/s, done.
Resolving deltas: 100% (47/47), done.
Checking connectivity... done.
frenetic@ubuntu-1404:~$ cd manual/programmers_guide
frenetic@ubuntu-1404:~/manual/programmers_guide$ cd code/quick_start
\end{minted}

The following code is in \codefilename{quick_start/repeater.py}:

\inputminted[linenos]{python}{code/quick_start/repeater.py}

Lines 1-2 are pretty much the same in every Frenetic network application.
Similarly, lines 15-16 are similar in most cases. 
 The meat of the application is an object class named RepeaterApp, whose base class is \netkat{frenetic.App}.
A frenetic application can hook code into different points of the network event cycle.
In our Repeater network app, the only two events we're interested in here are
\netkat{connected}, which is fired when a switch connects for the first time to Frenetic, and 
 \netkat{packet_in}, which is fired every time a packet
bound for a controller arrives.

The code in \netkat{connected} is called a \emph{handler} and here it 
merely directs the switch to send all packets to our application.
The code in \netkat{packet_in} is also a handler, and here it implements a {\it repeater}.
A repeater, sometimes called a {\it hub}, is the oldest type of network device. 
In a 2-port repeater, if a packet enters on port 1, it should get copied out to port 2.  
Conversely, if a packet enters on port 2, it should get copied out to port 1.
If there were more ports in our switch, we'd write a more sophisticated repeater -- one that
outputs the packet to all ports except the one on which it arrived (called the {\it ingress port}). 
We'll do that in section \ref{netkat_principles:efficient_sdn} 

\netkat{pkt_out} is a method provided by Frenetic to actually send the packet out the switch.  It takes three 
parameters: a switch, a packet, and a policy.  
Here the policy sends the packet out to port \netkat{out_port_id}.   

\section{Running The Repeater Application}

So let's get this running in a lab setup.  
Three programs need to be running:  Mininet, Frenetic, and our new Repeater application.  
For now, we'll run them in three separate command lines, having typed \netkat{vagrant ssh} in each
to login to the VM.

In the first terminal window, we'll start up Frenetic:

\begin{minted}{console}
frenetic@ubuntu-1404:~src/frenetic$ ./frenetic.native http-controller \
  > --verbosity debug
 [INFO] Calling create!
 [INFO] Current uid: 1000
 [INFO] Successfully launched OpenFlow controller with pid 3062
 [INFO] Connecting to first OpenFlow server socket
 [INFO] Failed to open socket to OpenFlow server: (Unix.Unix_error...
 [INFO] Retrying in 1 second
 [INFO] Successfully connected to first OpenFlow server socket
 [INFO] Connecting to second OpenFlow server socket
 [INFO] Successfully connected to second OpenFlow server socket 
\end{minted}

In the second, we'll start up Mininet with the same configuration as before:

\begin{minted}{console}
frenetic@ubuntu-1404:~$ sudo mn --topo=single,2 --controller=remote
*** Creating network
*** Adding controller
\end{minted}

The following will appear in your Frenetic window to show a connection has been made:

\begin{minted}{console}
 [INFO] switch 1 connected
[DEBUG] Setting up flow table
+----------------------+
| 1 | Pattern | Action |
|----------------------|
|             |        |
+----------------------+
\end{minted}

And in the third, we'll start our repeater application:

\begin{minted}{console}
frenetic@ubuntu-1404:~/manual/code/quick_start$ python repeater.py
Starting the tornado event loop (does not return).
\end{minted}

The following will appear in the Frenetic window.  

\begin{minted}{console}
 [INFO] GET /version
 [INFO] POST /quick_start/update_json
 [INFO] GET /quick_start/event
 [INFO] New client quick_start
[DEBUG] Installing policy
drop | port := pipe(repeater_app) | port := pipe(repeater_app)
[DEBUG] Setting up flow table
+---------------------------------------+
| 1 | Pattern | Action                  |
|---------------------------------------|
|             | Output(Controller(128)) |
+---------------------------------------+
\end{minted}

And finally, we'll pop over to the Mininet window and try our connection test once more:

\begin{minted}{console}
mininet> h1 ping h2
PING 10.0.0.2 (10.0.0.2) 56(84) bytes of data.
64 bytes from 10.0.0.2: icmp_seq=1 ttl=64 time=149 ms
64 bytes from 10.0.0.2: icmp_seq=2 ttl=64 time=97.2 ms
64 bytes from 10.0.0.2: icmp_seq=3 ttl=64 time=88.7 ms
\end{minted}

Ah, much better! 
Our pings are getting through.  
You can see evidence of this in the Frenetic window:

\begin{minted}{console}
 [INFO] GET /quick_start/event
 [INFO] POST /pkt_out
[DEBUG] SENDING PKT_OUT
 [INFO] GET /quick_start/event
 [INFO] POST /pkt_out
[DEBUG] SENDING PKT_OUT
 [INFO] GET /quick_start/event
 [INFO] POST /pkt_out
[DEBUG] SENDING PKT_OUT
\end{minted}

\section{Summary}

You now have a working SDN, or Software Defined Network!   Like much software, it works in layers:

\begin{figure}[h]
\centering
\includegraphics{frenetic_architecture}
\end{figure}

\begin{enumerate}
\item At the bottom is your switches and wires.  In our lab setup, Mininet and OpenVSwitch is a substitute for this layer.
\item In the middle is Frenetic.  It talks the OpenFlow protocol to the switches (or to Mininet) -- this is called the Southbound interface.  It also accepts its own language called NetKAT from network applications -- this is called the Northbound interface.
\item At the very top is your network application, which you write in Python.  It defines how packets are dealt with.
\end{enumerate}

We wrote a very simple network application that emulates a network repeater.   
It responds to the \netkat{packet_in} event coming from the switches through Frenetic when a packet 
arrives at the switch.  
And it sends the \netkat{pkt_out} message to send the packet back out through Frenetic to the switch.  
When you're done, your network application can be deployed to a real production network.

Obviously you can do much more than just simple repeating with SDN!  
We'll cover that next with some background on OpenFlow and NetKAT, the underlying language of Frenetic. 