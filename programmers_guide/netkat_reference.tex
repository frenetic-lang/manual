% !TEX root = frenetic_programmers_guide.tex

\chapter{Frenetic/NetKAT Reference}

\section{NetKAT Predicates}
\label{netkat_reference:predicates}

\subsection{Primitives}

\subsubsection{Drop}

\bigskip
\begin{tabularx}{\linewidth}{lX}
\textsc{Python}   & \texttt{Drop} \\
   & \texttt{false} \\ \\
\textsc{Raw}    & \texttt{false}  \\ \\
\textsc{REST} & \texttt{ \{ "type": "false" \} } 
\end{tabularx}

\python{Drop} matches no packets.  This is a \emph{predicate}, as opposed to the lower-case 
equivalent \python{drop} 
which is a \emph{policy} that drops all packets. 

Note in Python, \python{Id} and \python{Drop} are the only predicates that don't require parantheses.

\subsubsection{EthDst}

\bigskip
\begin{tabularx}{\linewidth}{lXX}
\textsc{Python}   & \texttt{EthDstEq("72:00:08:bc:5f:a0")}    & \texttt{EthDstNotEq("72:00:08:bc:5f:a0")} \\
    & \texttt{EthDstEq("72:00:08:bc:5f:a0", "82:9b:41:a6:16:f8")}  & \texttt{EthDstNotEq("72:00:08:bc:5f:a0", "82:9b:41:a6:16:f8")} \\
    & \texttt{EthDstEq(["72:00:08:bc:5f:a0", "82:9b:41:a6:16:f8"])}  & \texttt{EthDstNotEq(["72:00:08:bc:5f:a0", "82:9b:41:a6:16:f8"])} \\ \\
\textsc{Raw}    & \texttt{ethDst = 125344472129440}     & \\ \\
\textsc{REST} & \multicolumn{2}{l}{\texttt{ \{ "type": "test", "header": "ethdst", "value": 125344472129440 \} }} 
\end{tabularx}

\python{EthDstEq} matches packets with a particular Ethernet MAC destination address, or from a set of addresses -- 
if the 
parameter is a list of addresses, there is an implicit OR between them.    

In Python, a MAC address must be specified in colon-separated 6 byte hexadecimal.  This is the most common format
for MAC address display on network devices (although Cisco tends to list them in dotted notation with 2 byte
boundaries).  They must be passed as strings.  In Raw or REST-based NetKAT, you must send the 48-bit MAC as 
an integer.  

\subsubsection{EthSrc}

\bigskip
\begin{tabularx}{\linewidth}{lXX}
\textsc{Python}   & \texttt{EthSrcEq("72:00:08:bc:5f:a0")}    & \texttt{EthSrcNotEq("72:00:08:bc:5f:a0")} \\
    & \texttt{EthSrcEq("72:00:08:bc:5f:a0", "82:9b:41:a6:16:f8")}  & \texttt{EthSrcNotEq("72:00:08:bc:5f:a0", "82:9b:41:a6:16:f8")} \\
    & \texttt{EthSrcEq(["72:00:08:bc:5f:a0", "82:9b:41:a6:16:f8"])}  & \texttt{EthSrcNotEq(["72:00:08:bc:5f:a0", "82:9b:41:a6:16:f8"])} \\ \\
\textsc{Raw}    & \texttt{ethSrc = 125344472129440}     & \\ \\
\textsc{REST} & \multicolumn{2}{l}{\texttt{ \{ "type": "test", "header": "ethsrc", "value": 125344472129440 \} }} 
\end{tabularx}

\python{EthSrcEq} matches packets with a particular Ethernet MAC source address, or from a set of addresses -- 
if the 
parameter is a list of addresses, there is an implicit OR between them.    

In Python, a MAC address must be specified in colon-separated 6 byte hexadecimal.  This is the most common format
for MAC address display on network devices (although Cisco tends to list them in dotted notation with 2 byte
boundaries).  They must be passed as strings.  In Raw or REST-based NetKAT, you must send the 48-bit MAC as 
an integer.  

\subsubsection{EthType}

\bigskip
\begin{tabularx}{\linewidth}{lXX}
\textsc{Python}   & \texttt{EthTypeEq(0x800)}    & \texttt{EthTypeNotEq(0x800)} \\
    & \texttt{EthTypeEq(0x800, 0x806)}  & \texttt{EthTypeNotEq(0x800, 0x806)} \\
    & \texttt{EthTypeEq([0x800, 0x806])}  & \texttt{EthTypeNotEq([0x800, 0x806])} \\ \\
\textsc{Raw}    & \texttt{ethTyp = 2048}     & \\ \\
\textsc{REST} & \multicolumn{2}{l}{\texttt{ \{ "type": "test", "header": "ethtype", "value": 2048 \} }} 
\end{tabularx}

\python{EthType} matches packets with a particular Ethernet frame type, or from a set of frame types -- if the
parameter is a list of types, there is an implicit OR between them.  The frame type is a 32 bit 
integer as defined by IEEE, and the common ones are listed at 
http://www.iana.org/assignments/ieee-802-numbers/ieee-802-numbers.xhtml

Python can accept any valid 32 bit integer representation for the frame type -- it's common to pass them
as hexadecimal values because they're easy to remember.  You can also use constants defined in 
\python{ryu.packet.ether_types}.  Raw and REST-based NetKAT requires them to be specified in
decimal.

Certain OpenFlow match rules have \emph{dependencies} on the Ethernet type.  
In other words, it requires certain Ethernet type matches to be specified when other fields are matched.  For 
example, if the IPv4 source address is matched, then an Ethernet Type match must also be specified as 
IP, e.g. ethernet type 0x800.  This is not required in NetKAT - the Frenetic compiler will 
automatically compile the correct dependencies for you. 

With VLAN tagged packets, OpenFlow matches the
type of the \emph{enclosed packet}.  So for example, an IP packet wrapped with a VLAN tag will match 
\python{EthType(0x800)} even though the actual Ethernet type of the entire packet is 0x8100 (the 
IEEE 802.1q Ethernet type for VLAN packets).  

\subsubsection{Id}

\bigskip
\begin{tabularx}{\linewidth}{lX}
\textsc{Python}   & \texttt{Id} \\
   & \texttt{true} \\ \\
\textsc{Raw}    & \texttt{true}  \\ \\
\textsc{REST} & \texttt{ \{ "type": "true" \} } 
\end{tabularx}

\python{Id} matches all packets.  This is a \emph{predicate}, as opposed to the lower-case 
equivalent \python{id} 
which is a \emph{policy} that accepts all packets. 

Note in Python, \python{Id} and \python{Drop} are the only predicates that don't require parantheses.

\subsubsection{IPDst}

\bigskip
\begin{tabularx}{\linewidth}{lXX}
\textsc{Python}   & \texttt{IPDstEq("192.168.57.100")}    & \texttt{IPDstNotEq("192.168.57.100")} \\
    & \texttt{IPDstEq("192.168.57.0", 24)}  & \texttt{IPDstNotEq("192.168.57.0", 24)} \\ \\
\textsc{Raw}    & \texttt{ip4Dst = 192.168.57.100}     & \texttt{ip4Dst = 192.168.57.0/24} \\ \\
\textsc{REST} & \multicolumn{2}{l}{\texttt{ \{ "type": "test", "header": "ip4dst",}} \\
   & \multicolumn{2}{l}{\texttt{  "value": \{"addr": "192.168.57.100", "mask": 32 \}} \}} \\
   & \multicolumn{2}{l}{\texttt{ \{ "type": "test", "header": "ip4dst",}} \\
   & \multicolumn{2}{l}{\texttt{  "value": \{"addr": "192.168.57.0", "mask": 24 \}} \}} \\
\end{tabularx}

\python{IPDst} matches packets with a particular IP destination address, or destination network.  It only matches
IP v4 packets, as per Openflow 1.0.  

\python{IPDstEq} cannot accept a list of values as other predicates can.  
Instead, you can specify a subnet -- a range
of IP addresses using a bit mask value.  This is identical to using CIDR format, so that 
\python{IPDstEq("192.168.57.0", 24)} matches IP addresses in subnet \python{192.168.57.0/24} -- e.g. 
IP addresses \python{192.168.57.1} to \python{192.168.57.255}.  Leaving out the subnet mask in Python 
or Raw NetKAT is equivalent to specifiying a mask of 32 - meaning all 32 bits of the IP address are used
in the match.  (The 32 must be explicitly specified in REST-based NetKAT).

\subsubsection{IPProto}

\bigskip
\begin{tabularx}{\linewidth}{lXX}
\textsc{Python}   & \texttt{IPProtoEq(6)}    & \texttt{IPProtoNotEq(6)} \\
    & \texttt{IPProtoEq(6,17)}  & \texttt{IPProtoNotEq(6,17)} \\
    & \texttt{IPProtoEq([6,17])}  & \texttt{IPProtoNotEq([6,17])} \\ \\
\textsc{Raw}    & \texttt{ipProto = 6}     & \\ \\
\textsc{REST} & \multicolumn{2}{l}{\texttt{ \{ "type": "test", "header": "ipproto", "value": 6 \} }} 
\end{tabularx}

\python{IPProtoEq} matches packets with a certain IP Protocol, or set of protocols -- if a list is specified, 
there is an implicit OR between them.  It only matches IP v4 packets, as per OpenFlow 1.0 specs.  

The IP Protocol is a number from 1-255.  A complete list of common protocols are listed in 
https://en.wikipedia.org/wiki/List\_of\_IP\_protocol\_numbers.  In Python, you can also match against the
constants in \python{ryu.packet.in_proto}.

\subsubsection{IPSrc}

\bigskip
\begin{tabularx}{\linewidth}{lXX}
\textsc{Python}   & \texttt{IPSrcEq("192.168.57.100")}    & \texttt{IPSrcNotEq("192.168.57.100")} \\
    & \texttt{IPSrcEq("192.168.57.0", 24)}  & \texttt{IPSrcNotEq("192.168.57.0", 24)} \\ \\
\textsc{Raw}    & \texttt{ip4Src = 192.168.57.100}     & \texttt{ip4Src = 192.168.57.0/24} \\ \\
\textsc{REST} & \multicolumn{2}{l}{\texttt{ \{ "type": "test", "header": "ip4src",}} \\
   & \multicolumn{2}{l}{\texttt{  "value": \{"addr": "192.168.57.100", "mask": 32 \}} \}} \\
   & \multicolumn{2}{l}{\texttt{ \{ "type": "test", "header": "ip4src",}} \\
   & \multicolumn{2}{l}{\texttt{  "value": \{"addr": "192.168.57.0", "mask": 24 \}} \}} \\
\end{tabularx}

\python{IPSrc} matches packets with a particular IP source address, or source network.  It only matches
IP v4 packets, as per Openflow 1.0.  

\python{IPSrcEq} cannot accept a list of values as other predicates can.  
Instead, you can specify a subnet -- a range
of IP addresses using a bit mask value.  This is identical to using CIDR format, so that 
\python{IPSrcEq("192.168.57.0", 24)} matches IP addresses in subnet \python{192.168.57.0/24} -- e.g. 
IP addresses \python{192.168.57.1} to \python{192.168.57.255}.  Leaving out the subnet mask in Python 
or Raw NetKAT is equivalent to specifiying a mask of 32 - meaning all 32 bits of the IP address are used
in the match.  (The 32 must be explicitly specified in REST-based NetKAT).


\subsubsection{Port}

\bigskip
\begin{tabularx}{\linewidth}{lXX}
\textsc{Python} 	& \texttt{PortEq(1)} 		& \texttt{PortNotEq(1)} \\
	 	& \texttt{PortEq(1,2,3)} 	& \texttt{PortNotEq(1,2,3)} \\
	 	& \texttt{PortEq([1,2,3])} 	& \texttt{PortNotEq([1,2,3])} \\ \\
\textsc{Raw}		& \texttt{port = 1} 		& \\ \\
\textsc{REST}	& \multicolumn{2}{l}{\texttt{ \{ "type": "test", "header": "port", "value": 1 \} }} 
\end{tabularx}

\python{PortEq} matches packets that arrive on a particular port, or on a particular set of ports -- if the 
parameter is a list of ports, there is an implicit OR between them.  

A port number must be an integer from
1 \ldots 65520, the upper limit of physical port numbers according to the OpenFlow 1.0 spec.  It can also be 
a string convertible to an integer in this range.  The port number is not checked against any known list
of port numbers on the switch, so you can insert rules for non-operational ports (they obviously won't
match any packets until the port becomes operational).  

\python{PortEq} predicates are usually combined with \python{SwitchEq} predicates because packet processing
is switch-and-port-specific.  However, it may be used alone if, for example, a packet on port 47 is processed exactly
the same on every switch (for example, port 47 is the trunk port between switches).  

\subsubsection{SwitchEq}

\bigskip
\begin{tabularx}{\linewidth}{lXX}
\textsc{Python} 	& \texttt{SwitchEq(1981745)} 		& \texttt{SwitchNotEq(1981745)} \\
	 	& \texttt{SwitchEq(1981745, 887345)} 	& \texttt{SwitchNotEq(1981745, 887345)} \\
	 	& \texttt{SwitchEq([1981745, 887345])} 	& \texttt{SwitchNotEq([1981745, 887345])} \\ \\
\textsc{Raw}		& \texttt{switch = 1981745} 		& \\ \\
\textsc{REST}	& \multicolumn{2}{l}{\texttt{ \{ "type": "test", "header": "switch", "value": 1981745 \} }} 
\end{tabularx}

\python{SwitchEq} matches packets that arrive on a particular switch or set of switches.  
Switches are identified by 
a 64-bit number called a \emph{datapath ID} or \emph{DPID}.  On OpenVSwitch, the default OpenFlow
switch provided by Mininet, DPID's are generally small integers corresponding to the switch name --
e.g. \python{s1} in Mininet has the DPID 1.  Physical switches generally append the OpenFlow instance
number with the overall MAC id of the switch so it's globally unique.  But this varies from switch
to switch. 

\python{SwitchEq} predicates are usually combined with \python{PortEq} predicates because packet processing
is switch-and-port-specific.  

\subsubsection{TcpDstPort}

\bigskip
\begin{tabularx}{\linewidth}{lXX}
\textsc{Python}   & \texttt{TcpDstPortEq(80)}    & \texttt{TcpDstPortNotEq(80)} \\
    & \texttt{TcpDstPortEq(80,443)}  & \texttt{TcpDstPortNotEq(80,443)} \\
    & \texttt{TcpDstPortEq([80,443])}  & \texttt{TcpDstPortNotEq([80,443])} \\ \\
\textsc{Raw}    & \texttt{tcpDstPort = 80}     & \\ \\
\textsc{REST} & \multicolumn{2}{l}{\texttt{ \{ "type": "test", "header": "tcpdstport", "value": 80 \} }} 
\end{tabularx}

\python{TcpDstPortEq} matches packets with a certain TCP or UDP destination port, or set of ports -- if the 
parameter is a list of ports, there is an implicit OR between them.  Note this only matches TCP or UDP packets,
which must be IP packets.  A complete list of common destination ports are listed in 
https://en.wikipedia.org/wiki/List\_of\_TCP\_and\_UDP\_port\_numbers

\subsubsection{TcpSrcPort}

\bigskip
\begin{tabularx}{\linewidth}{lXX}
\textsc{Python}   & \texttt{TcpSrcPortEq(5000)}    & \texttt{TcpSrcPortNotEq(5000)} \\
    & \texttt{TcpSrcPortEq(5000,5001)}  & \texttt{TcpSrcPortNotEq(5000,5001)} \\
    & \texttt{TcpSrcPortEq([5000,5001])}  & \texttt{TcpSrcPortNotEq([5000,5001])} \\ \\
\textsc{Raw}    & \texttt{tcpSrcPort = 5000}     & \\ \\
\textsc{REST} & \multicolumn{2}{l}{\texttt{ \{ "type": "test", "header": "tcpsrcport", "value": 5000 \} }} 
\end{tabularx}

\python{TcpSrcPortEq} matches packets with a certain TCP or UDP source port, or set of ports -- if the 
parameter is a list of ports, there is an implicit OR between them.  Note this only matches TCP or UDP packets,
which must be IP packets.  

\python{TcpSrcPort} matches are rarely used in OpenFlow since TCP and UDP source ports are essentially random
numbers assigned by the client.  They can be used for ``per-flow'' rules where a rule lasts for the 
duration of a conversation between a client and a server.  

\subsubsection{Vlan}

\bigskip
\begin{tabularx}{\linewidth}{lXX}
\textsc{Python}   & \texttt{VlanEq(1001)}    & \texttt{VlanNotEq(1001)} \\
    & \texttt{VlanEq(1001,1002)}  & \texttt{VlanNotEq(1001,1002)} \\
    & \texttt{VlanEq([1001,1002])}  & \texttt{VlanNotEq([1001,1002])} \\ \\
\textsc{Raw}    & \texttt{vlanId = 1001}     & \\ \\
\textsc{REST} & \multicolumn{2}{l}{\texttt{ \{ "type": "test", "header": "vlan", "value": 1001 \} }} 
\end{tabularx}

\python{VlanEq} matches packets that arrive with a particular VLAN tag, or on a particular set of VLAN tags -- 
if the 
parameter is a list of vlans, there is an implicit OR between them.  

The VLAN id, as defined by IEE 802.1q is a 12 bit value from 1 \ldots 4095.  VLAN matches are only applicable
to packets with an actual VLAN tag, so the Ethernet Type is 0x8100.  However, in such cases, 
OpenFlow exposes the Ethernet 
Type of the \emph{enclosed packet}.  So for example, an IP packet wrapped with a VLAN tag will match 
\python{EthType(0x800)}.  

In most physical switches, access ports are tagged with a particular VLAN.  So though the packet coming
from the host is untagged, the tagging occurs at the switch before the OpenFlow engine is invoked.  
That way you can assign access ports to VLANs, and any packets coming in on that port will match the
appropriate \python{VlanEq}.  This is one of the few cases where packet headers are manipulated outside the
OpenFlow engine.  

\subsubsection{VlanPcp}

\bigskip
\begin{tabularx}{\linewidth}{lXX}
\textsc{Python}   & \texttt{VlanPcpEq(1)}    & \texttt{VlanPcpNotEq(1)} \\
    & \texttt{VlanPcpEq(1,2)}  & \texttt{VlanPcpNotEq(1,2)} \\
    & \texttt{VlanPcpEq([1,2])}  & \texttt{VlanPcpNotEq([1,2])} \\ \\
\textsc{Raw}    & \texttt{vlanPcp = 1}     & \\ \\
\textsc{REST} & \multicolumn{2}{l}{\texttt{ \{ "type": "test", "header": "vlanpcp", value: 1 \} }} 
\end{tabularx}

\python{VlanPcpEq} matches packets that arrive with a particular VLAN Priority Code Point (PCP), 
or on a particular set of VLAN PCP's -- 
if the 
parameter is a list of PCP's, there is an implicit OR between them.  Only VLAN tagged packets
will match this predicate.

The PCP as defined by IEEE 802.1q must be an integer from 0-7.  0 is the default, generally meaning
best effort delivery.  

\subsection{Combinations}

\subsubsection{And}

\bigskip
\begin{tabularx}{\linewidth}{lX}
\textsc{Python}   & \texttt{$pred_1$ \& $pred_2$} \\
    & \texttt{And($pred_1$, $pred_2$, \ldots, $pred_n$)} \\ \\
\textsc{Raw}    & \texttt{$pred_1$ and $pred_2$}  \\ \\
\textsc{REST} & \texttt{ \{ "type": "and", "preds": [ $pred_1$, $pred_2$, \ldots $pred_n$ ] \} } 
\end{tabularx}

\python{And} is the Boolean conjunction of predicates. 

\subsubsection{Not}

\bigskip
\begin{tabularx}{\linewidth}{lX}
\textsc{Python}   & \texttt{\textasciitilde $pred_1$} \\
    & \texttt{Not($pred_1$)} \\ \\
\textsc{Raw}    & \texttt{not $pred_1$}  \\ \\
\textsc{REST} & \texttt{ \{ "type": "neg", "pred": $pred_1$ \} } 
\end{tabularx}

\python{Not} is the Boolean negation of a single predicate.

\subsubsection{Or}

\bigskip
\begin{tabularx}{\linewidth}{lX}
\textsc{Python}   & \texttt{$pred_1$ $\vert$ $pred_2$} \\
    & \texttt{Or($pred_1$, $pred_2$, \ldots, $pred_n$)} \\ \\
\textsc{Raw}    & \texttt{$pred_1$ or $pred_2$}  \\ \\
\textsc{REST} & \texttt{ \{ "type": "or", "preds": [ $pred_1$, $pred_2$, \ldots $pred_n$ ] \} } \end{tabularx}

\python{Or} is the Boolean disjunction of predicates.  In Python, if all of your disjunction terms involve
the same field, you can use the list form of the simple predicate as a shortcut (except for 
\python{IpSrcEq} and \python{IpDstEq}). 

Note that in Python \python{Or} shares a symbol with \python{Union}, but the parser keeps the meaning straight.

\section{Policies}

\subsubsection{drop}

\bigskip
\begin{tabularx}{\linewidth}{lX}
\textsc{Python}   & \texttt{drop} \\ \\
\textsc{Raw}   & \texttt{drop} \\ \\
\textsc{REST} & \texttt{ \{ "type": "filter", "pred": \{ "type": false \} \} } 
\end{tabularx}

\python{drop} is a \emph{policy} that drops all packets.  It is equivalent to \python{Filter(false)}.

Note that in Python, \python{id} and \python{drop} are the only policies that begin with a lower case
letter, and do not require parantheses.  

\subsubsection{Filter}

\bigskip
\begin{tabularx}{\linewidth}{lX}
\textsc{Python}   & \texttt{Filter($pred$)} \\ \\
\textsc{Raw}   & \texttt{filter $pred$} \\ \\
\textsc{REST} & \texttt{ \{ "type": "filter", "pred": $pred$ \} } 
\end{tabularx}

\python{Filter} accepts all packets that match the predicate $pred$ and rejects everything else.
It is described in detail in Section \ref{section:combining}.

\subsubsection{id}

\bigskip
\begin{tabularx}{\linewidth}{lX}
\textsc{Python}   & \texttt{id} \\ \\
\textsc{Raw}   & \texttt{id} \\ \\
\textsc{REST} & \texttt{ \{ "type": "filter", "pred": \{ "type": true \} \} } 
\end{tabularx}

\python{drop} is a \emph{policy} that drops all packets.  It is equivalent to \python{Filter(true)}.

Note that in Python, \python{id} and \python{drop} are the only policies that begin with a lower case
letter, and do not require parantheses.  

\subsubsection{SetEthDst}

\bigskip
\begin{tabularx}{\linewidth}{lX}
\textsc{Python}   & \texttt{SetEthDst("72:00:08:bc:5f:a0")} \\ \\
\textsc{Raw}    & \texttt{ethDst := 125344472129440}     \\ \\
\textsc{REST} & \texttt{ \{ "type": "mod", "header": "ethdst", "value": 125344472129440 \} }
\end{tabularx}

\python{SetEthDst} sets the Ethernet MAC destination address for a packet.    

In Python, a MAC address must be specified in colon-separated 6 byte hexadecimal.  This is the most common format
for MAC address display on network devices (although Cisco tends to list them in dotted notation with 2 byte
boundaries).  They must be passed as strings.  In Raw or REST-based NetKAT, you must send the 48-bit MAC as 
an integer.  

\subsubsection{SetEthSrc}

\bigskip
\begin{tabularx}{\linewidth}{lX}
\textsc{Python}   & \texttt{SetEthSrc("72:00:08:bc:5f:a0")} \\ \\
\textsc{Raw}    & \texttt{ethSrc := 125344472129440}     \\ \\
\textsc{REST} & \texttt{ \{ "type": "mod", "header": "ethsrc", "value": 125344472129440 \} }
\end{tabularx}

\python{SetEthDst} sets the Ethernet MAC source address for a packet.    

In Python, a MAC address must be specified in colon-separated 6 byte hexadecimal.  This is the most common format
for MAC address display on network devices (although Cisco tends to list them in dotted notation with 2 byte
boundaries).  They must be passed as strings.  In Raw or REST-based NetKAT, you must send the 48-bit MAC as 
an integer.  

\subsubsection{SetEthType}

\bigskip
\begin{tabularx}{\linewidth}{lX}
\textsc{Python}   & \texttt{SetEthType(0x800)} \\ \\
\textsc{Raw}    & \texttt{ethTyp := 2048}     \\ \\
\textsc{REST} & \texttt{ \{ "type": "mod", "header": "ethtype", "value": 2048 \} }
\end{tabularx}

\python{SetEthType} sets the Ethernet type for a packet.    

Note that for VLAN-tagged packets, you set the Ethernet type for the \emph{inner} packet, not 0x8100
for VLAN.  To tag it with a VLAN, you simply use the policy \netkat{SetVlan}, and that assigns the
correct Ethernet type to the outer packet.  

\subsubsection{SetIPDst}

\bigskip
\begin{tabularx}{\linewidth}{lX}
\textsc{Python}   & \texttt{SetIPDst("192.168.57.100")} \\ \\
\textsc{Raw}    & \texttt{ip4Dst := 192.168.57.100}     \\ \\
\textsc{REST} & \texttt{ \{ "type": "mod", "header": "ip4dst", } \\
  & \texttt{"value": \{"addr": "192.168.57.100" \} \} }
\end{tabularx}

\python{SetIPDst} sets the IP v4 destination address.  Note that unlike the \netkat{IPDstEq} predicate,
there is no option for masking.  You must set a fixed IP address with all bits present.    

The proper Ethernet type for IP packets must already be set in the incoming packet, or through the 
\netkat{SetEthType} policy.  Otherwise, the packet will likely be dropped by a host or router before it 
reaches its destination.

\subsubsection{SetIPProto}

\bigskip
\begin{tabularx}{\linewidth}{lX}
\textsc{Python}   & \texttt{SetIPProto(6)} \\ \\
\textsc{Raw}    & \texttt{ipProto := 6}     \\ \\
\textsc{REST} & \texttt{ \{ "type": "mod", "header": "ipProto", "value": 6 \} }
\end{tabularx}

\python{SetIPProto} sets the IP v4 protocol type.      

The proper Ethernet type for IP packets must already be set in the incoming packet, or through the 
\netkat{SetEthType} policy.  Otherwise, the packet will likely be dropped by a host or router before it 
reaches its destination.

\subsubsection{SetIPSrc}

\bigskip
\begin{tabularx}{\linewidth}{lX}
\textsc{Python}   & \texttt{SetIPSrc("192.168.57.100")} \\ \\
\textsc{Raw}    & \texttt{ip4Src := 192.168.57.100}     \\ \\
\textsc{REST} & \texttt{ \{ "type": "mod", "header": "ip4src", } \\
  & \texttt{"value": \{"addr": "192.168.57.100" \} \} }
\end{tabularx}

\python{SetIPSrc} sets the IP v4 destination address.  Note that unlike the \netkat{IPSrcEq} predicate,
there is no option for masking.  You must set a fixed IP address with all bits present.    

The proper Ethernet type for IP packets must already be set in the incoming packet, or through the 
\netkat{SetEthType} policy.  Otherwise, the packet will likely be dropped by a host or router before it 
reaches its destination.

\subsubsection{SetPort}

\bigskip
\begin{tabularx}{\linewidth}{lX}
\textsc{Python}   & \texttt{SetPort(1)} \\ \\
\textsc{Raw}    & \texttt{port := 1}     \\ \\
\textsc{REST} & \texttt{ \{ "type": "mod", "header": "port", "value": 1 \} }
\end{tabularx}

\python{SetPort} sets the port destination for this packet.      

A port number must be an integer from
1 \ldots 65520, the upper limit of physical port numbers according to the OpenFlow 1.0 spec.  It can also be 
a string convertible to an integer in this range.  The port number is not checked against any known list
of port numbers on the switch, but any attempt to send a packet over a non-existent or non-operational
port will fail.  

\section{Events}

\section{Commands}

\section{Compiler Directives}

\section{Frenetic Command Line}

